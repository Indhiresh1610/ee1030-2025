\documentclass[journal]{IEEEtran}
\usepackage[a5paper, margin=10mm]{geometry}
%\usepackage{lmodern} % Ensure lmodern is loaded for pdflatex
\usepackage{tfrupee} % Include tfrupee package


\setlength{\headheight}{1cm} % Set the height of the header box
\setlength{\headsep}{0mm}     % Set the distance between the header box and the top of the text


%\usepackage[a5paper, top=10mm, bottom=10mm, left=10mm, right=10mm]{geometry}

%
\setlength{\intextsep}{10pt} % Space between text and floats

\makeindex


\usepackage{cite}
\usepackage{amsmath,amssymb,amsfonts,amsthm}
\usepackage{algorithmic}
\usepackage{graphicx}
\usepackage{textcomp}
\usepackage{xcolor}
\usepackage{txfonts}
\usepackage{listings}
\usepackage{enumitem}
\usepackage{mathtools}
\usepackage{gensymb}
\usepackage{comment}
\usepackage[breaklinks=true]{hyperref}
\usepackage{tkz-euclide} 
\usepackage{listings}
\usepackage{multicol}
\usepackage{xparse}
\usepackage{gvv}
%\def\inputGnumericTable{}                                 
\usepackage[latin1]{inputenc}                                
\usepackage{color}                                            
\usepackage{array}                                            
\usepackage{longtable}                                       
\usepackage{calc}                                             
\usepackage{multirow}                                         
\usepackage{hhline}                                           
\usepackage{ifthen}                                               
\usepackage{lscape}
\usepackage{tabularx}
\usepackage{array}
\usepackage{float}
\usepackage{ar}
\usepackage[version=4]{mhchem}


\newtheorem{theorem}{Theorem}[section]
\newtheorem{problem}{Problem}
\newtheorem{proposition}{Proposition}[section]
\newtheorem{lemma}{Lemma}[section]
\newtheorem{corollary}[theorem]{Corollary}
\newtheorem{example}{Example}[section]
\newtheorem{definition}[problem]{Definition}
\newcommand{\BEQA}{\begin{eqnarray}}
\newcommand{\EEQA}{\end{eqnarray}}

\theoremstyle{remark}


\begin{document}
\bibliographystyle{IEEEtran}
\onecolumn

\title{12.665}
\author{INDHIRESH S- EE25BTECH11027}
\maketitle


\renewcommand{\thefigure}{\theenumi}
\renewcommand{\thetable}{\theenumi}

\textbf{Question}.The product of eigenvalues of
  \begin{align*}
      \myvec{0&0&1\\0&1&0\\1&0&0}
  \end{align*}
\begin{enumerate}
    \item -1
    \item 1
    \item  0
    \item 2
\end{enumerate}
\textbf{Solution}:
Let
\begin{align}
 \Vec{A}=\myvec{0&0&1\\0&1&0\\1&0&0}
\end{align}
Let $\lambda$ be the eigen value of the $\Vec{A}$. Then,

\begin{align}
  \mydet{\Vec{A}-\lambda\Vec{I}}=0
\end{align}

\begin{align}
  \mydet{-\lambda&0&1\\0&1-\lambda&0\\1&0&-\lambda}=0
\end{align}

\begin{align}
  -\lambda(1-\lambda)(-\lambda)+1(-(1-\lambda))=0
\end{align}

\begin{align}
\lambda^2(1-\lambda)-(1-\lambda)=0
\end{align}
\begin{align}
    (\lambda^2-1)(1-\lambda)=0
\end{align}
\begin{align}
    \lambda=1\;\;and\;\;\lambda=-1
\end{align}
Product of two eigen values is -1



\end{document}