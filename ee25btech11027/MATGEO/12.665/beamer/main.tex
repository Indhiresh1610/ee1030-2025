\documentclass{beamer}
    \usepackage[utf8]{inputenc}
    
    \usetheme{Madrid}
    \usecolortheme{default}
    \usepackage{amsmath,amssymb,amsfonts,amsthm}
    \usepackage{mathtools}
    \usepackage{txfonts}
    \usepackage{tkz-euclide}
    \usepackage{listings}
    \usepackage{adjustbox}
    \usepackage{array}
    \usepackage{gensymb}
    \usepackage{tabularx}
    \usepackage{gvv}
    \usepackage{lmodern}
    \usepackage{circuitikz}
    \usepackage{tikz}
    \lstset{literate={·}{{$\cdot$}}1 {λ}{{$\lambda$}}1 {→}{{$\to$}}1}
    \usepackage{graphicx}
    
    \setbeamertemplate{page number in head/foot}[totalframenumber]
    
    \usepackage{tcolorbox}
    \tcbuselibrary{minted,breakable,xparse,skins}
    
    
    
    \definecolor{bg}{gray}{0.95}
    \DeclareTCBListing{mintedbox}{O{}m!O{}}{%
      breakable=true,
      listing engine=minted,
      listing only,
      minted language=#2,
      minted style=default,
      minted options={%
        linenos,
        gobble=0,
        breaklines=true,
        breakafter=,,
        fontsize=\small,
        numbersep=8pt,
        #1},
      boxsep=0pt,
      left skip=0pt,
      right skip=0pt,
      left=25pt,
      right=0pt,
      top=3pt,
      bottom=3pt,
      arc=5pt,
      leftrule=0pt,
      rightrule=0pt,
      bottomrule=2pt,
      toprule=2pt,
      colback=bg,
      colframe=orange!70,
      enhanced,
      overlay={%
        \begin{tcbclipinterior}
        \fill[orange!20!white] (frame.south west) rectangle ([xshift=20pt]frame.north west);
        \end{tcbclipinterior}},
      #3,
    }
    \lstset{
        language=C,
        basicstyle=\ttfamily\small,
        keywordstyle=\color{blue},
        stringstyle=\color{orange},
        commentstyle=\color{green!60!black},
        numbers=left,
        numberstyle=\tiny\color{gray},
        breaklines=true,
        showstringspaces=false,
    }
    %------------------------------------------------------------
    %This block of code defines the information to appear in the
    %Title page
    \title %optional
    {12.665}
    \date{17 October, 2025}
    %\subtitle{A short story}
    
    \author % (optional)
    {INDHIRESH S - EE25BTECH11027}
    
    \begin{document}
    
    \frame{\titlepage}
    
    \begin{frame}{Question}
 The product of eigenvalues of
  \begin{align*}
      \myvec{0&0&1\\0&1&0\\1&0&0}
  \end{align*}
\begin{enumerate}
    \item -1
    \item 1
    \item  0
    \item 2
\end{enumerate}
    \end{frame}
    
    \begin{frame}[allowframebreaks] 
    \frametitle{Equation}
        \centering
        \label{tab:parameters}
 Let
\begin{align}
 \Vec{A}=\myvec{0&0&1\\0&1&0\\1&0&0}
\end{align}
Let $\lambda$ be the eigen value of the $\Vec{A}$. Then,

\begin{align}
  \mydet{\Vec{A}-\lambda\Vec{I}}=0
\end{align}

    \end{frame}
    
    \begin{frame}
    \frametitle{Theoretical Solution}
   \begin{align}
  \mydet{-\lambda&0&1\\0&1-\lambda&0\\1&0&-\lambda}=0
\end{align}

\begin{align}
  -\lambda(1-\lambda)(-\lambda)+1(-(1-\lambda))=0
\end{align}

\begin{align}
\lambda^2(1-\lambda)-(1-\lambda)=0
\end{align}
\begin{align}
    (\lambda^2-1)(1-\lambda)=0
\end{align}
\begin{align}
    \lambda=1\;\;and\;\;\lambda=-1
\end{align}
Product of two eigen values is -1



    \end{frame}
    
    

    \end{document}
